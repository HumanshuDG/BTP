\section{Introduction}

\subsection{Motivation}
India being majorly agrarian economy, farmers' livelihoods are greatly impacted by the pricing of crops like cotton and oranges. India is one of the largest producers and exporters of cotton and oranges in the world. These crops' prices are frequently unstable and change depending on a number of variables, including the weather, supply and demand, and governmental policy. Farmers and traders find it challenging to plan the timing of their product sales and purchases in light of these changes. Accurate crop price forecasts can assist farmers, dealers, and policymakers in making educated decisions. This issue is made worse by the absence of an accurate forecasting model for crop prices. Machine learning algorithms can be used to estimate future prices by analyzing past price trends and other pertinent data.

\subsection{Background}
\subsubsection{Agricultural Significance in India}
The agricultural sector in India is the backbone of the economy, supporting the livelihoods of a substantial portion of the population. The pricing of key crops, such as oranges and cotton, not only impacts farmers directly but also influences the overall economic landscape. Understanding and predicting the fluctuations in crop prices is paramount for informed decision-making and sustainable agricultural practices.

\subsubsection{Dataset Overview}
The project leverages a comprehensive dataset provided by the Indian Government, offering a historical perspective on crop prices across diverse regions and seasons. This dataset serves as the bedrock for the development and validation of machine learning models for short-term forecasting.

\subsection{Exploratory Data Analysis (EDA)}
    \begin{enumerate}
        \item Data Cleaning and Imputation\\
            Prior to analysis, a meticulous EDA is conducted to address missing data and ensure the integrity of the dataset. Techniques such as imputation and data cleaning are employed to enhance the quality of the dataset, laying the groundwork for accurate and reliable forecasting models.
        \item Pattern Identification\\
            The EDA phase aims to uncover inherent patterns within the data. Through statistical analyses and visualization techniques, the project seeks to discern trends, cyclical patterns, and any anomalies that may be indicative of the underlying dynamics governing crop prices.
    \end{enumerate}

\subsection{Geographic Segmentation}
    \begin{enumerate}
        \item Urban and Rural Tiers
            Recognizing the diversity of the Indian agricultural landscape, the dataset is segmented into tiers representing urban and rural cities. This segmentation allows for a nuanced examination of how regional disparities influence pricing dynamics, providing valuable insights for targeted interventions.
        \item State-Level Analysis
            The project delves into state-level dynamics, acknowledging the varied agricultural practices and market conditions across different states in India. This analysis contributes to a more localized understanding of the factors shaping crop prices.
        \item City-Specific Insights
            Further granularity is achieved by focusing on specific cities, allowing the project to unravel localized trends and intricacies. This micro-level analysis facilitates a fine-tuned evaluation of model accuracy and aids in tailoring strategies to address city-specific challenges.
    \end{enumerate}

\subsection{Model Development and Selection}
    \begin{enumerate}
        \item Regression-Based Algorithms\\
            Four candidate algorithms—Artificial Neural Network (ANN), AutoRegressive Integrated Moving Average (ARIMA), Logistic Regression (LR), and Support Vector Regression (SVR)—are chosen for their capacity to handle regression tasks. The inclusion of ARIMA serves as a benchmark, offering a basis for comparison against machine learning approaches.
        \item Training-Testing-Validation Sets\\
            The dataset is strategically partitioned into training, testing, and validation sets to enable rigorous evaluation of model performance. This ensures that the selected algorithms are not only accurate in their predictions but also generalize well to unseen data.        
    \end{enumerate}

\subsection{Comparative Analysis}
    \begin{enumerate}
        \item National Overview\\
            The project begins with a broad national analysis, providing a holistic understanding of crop price forecasting across the entire country. This sets the stage for subsequent, more detailed examinations at lower geographic levels.
        \item State-Level Performance\\
            The effectiveness of the forecasting models is assessed at the state level, allowing for the identification of regional variations and the validation of model robustness in diverse agricultural contexts.
        \item City-Specific Accuracy\\
            The project concludes with an exploration of city-specific insights, shedding light on how well the models perform in distinct urban and rural environments. This localized evaluation is crucial for tailoring strategies to address specific market conditions.            
    \end{enumerate}
    In summary, this project is a comprehensive exploration of machine learning-based short-term forecasting for orange and cotton crop prices in the Indian market. By incorporating geographic segmentation and seasonal considerations, the aim is not only to provide accurate predictions but also to offer actionable insights for stakeholders at different levels of the agricultural supply chain. Through this multifaceted approach, the project contributes to the ongoing efforts to enhance the resilience and sustainability of India's agricultural sector.