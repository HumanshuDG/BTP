\section{Conclusion and Future Scope}

    \subsection{Conclusion}
        In conclusion, this project successfully establishes a Machine Learning-based short-term forecasting mechanism for orange and cotton crop prices in the Indian market. Leveraging a comprehensive dataset provided by the Indian Government, the study incorporates Exploratory Data Analysis (EDA) to address data challenges and reveal underlying patterns. The model development phase involves the careful selection of regression-based algorithms, including Artificial Neural Networks (ANN), Auto Regressive Integrated Moving Average (ARIMA), Logistic Regression (LR), and Support Vector Regression (SVR). Importantly, the inclusion of an ARIMA model serves as a benchmark for comparative analysis. The geographic segmentation of the dataset into urban and rural tiers, state-level analysis, and city-specific insights enriches the model with regional and seasonal dynamics, providing a nuanced understanding of pricing variations. The model's performance is evaluated at national, state, and city levels, offering a comprehensive assessment. This multifaceted approach not only contributes to accurate short-term price predictions but also furnishes actionable insights for stakeholders in the agricultural supply chain, thereby enhancing the resilience and sustainability of India's agricultural sector.
    
    \subsection{Future Scope}
        \begin{enumerate}
            \item
                Integration of Additional Factors: Future iterations of the model could consider incorporating additional factors such as climate data, market demand, and government policies to enhance predictive accuracy and comprehensiveness.
            
            \item
                Dynamic Model Updating: The development of a dynamic updating mechanism would ensure that the model adapts to changing market conditions, further improving its applicability and relevance over time.
            \item
                Extension to Other Crops: The methodology developed in this project can be extended to other key crops, broadening the scope of the model's applicability and providing a more holistic understanding of India's agricultural pricing dynamics.
            \item
                Incorporation of Advanced ML Techniques: The integration of advanced machine learning techniques, such as ensemble models or deep learning architectures, could be explored to push the boundaries of forecasting precision.
            \item
                Stakeholder Engagement: Future research could involve greater stakeholder engagement, ensuring that the developed model aligns with the practical needs and perspectives of those directly involved in the agricultural sector.
            \item
                Real-time Predictions: Building real-time prediction capabilities would be a valuable enhancement, allowing stakeholders to access up-to-the-minute insights for more agile decision-making.
        \end{enumerate}