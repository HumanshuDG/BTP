\section{Proposed Methodology}
% block styles
    \tikzstyle{block} = [rectangle, draw, fill = blue!20, text width = 11em, text centered, rounded corners, minimum height = 2em]
    \tikzstyle{mlmodel} = [rectangle, fill = blue!8, draw = white, text width = 7em, text centered, rounded corners, font = \scriptsize]
    \tikzstyle{mlmodellside} = [rectangle, fill = blue!14, text width = 6em, text height = 4em, text centered, rounded corners, font = \scriptsize]
    \tikzstyle{mlmodelrside} = [rectangle, fill = blue!14, text centered, text width = 30em, text height = 4em, rounded corners, font = \tiny]
    \tikzstyle{mlmodelrsidetop} = [rectangle, text centered, align = north, text height = 2em, rounded corners, font = \tiny]
    \tikzstyle{mlmodelrsidebottom} = [rectangle, text centered, align = center, text width = 50 em, text height = 2em, rounded corners, font = \tiny]
    \tikzstyle{mlmodelrsidenode} = [rectangle, fill = blue!8, draw = white, text centered, text width = 7em, rounded corners, font = \scriptsize]
    \tikzstyle{level} = [rectangle, text centered, text width = 34em, rounded corners, font = \tiny]
    \tikzstyle{levell} = [rectangle, text centered, text width = 34em,, height = 30em, rounded corners, font = \tiny]
    \tikzstyle{line} = [draw, -latex']
    \tikzstyle{cloud} = [draw, ellipse, fill = red!20, node distance = 12em, minimum height = 2em]
    
    \begin{figure}[H]
        \centering \sffamily
        \resizebox{\textwidth}{!}{
            \begin{tikzpicture}[node distance = 1 em]
            
                % define nodes
                \node[circle, draw, font=\tiny] (start) {START};
                \node [block, below = of start] (data) {Data Collection \\ {\scriptsize{data.gov.in \& agmarknet.gov.in}}};
                \node [block, below = of data] (preprocess) {Data Preprocessing};
                \node [block, below = of preprocess] (features) {Feature Engineering};
                \node [block, below = of features, text width = 38em] (mlmodel) {Selecting Machine Learning Model (using candidate algorithms) \\
                    {
                        \begin{tikzpicture}
                            \node[mlmodellside] (mlmodellside) {
                                \begin{tikzpicture}[node distance = 0.5em]
                                    \node [mlmodel] (training) {Training};
                                    \node [mlmodel, below = 1.4em of training] (validation) {Validation};
                                    \path [line] ([xshift=2em]  validation.north)  --([xshift=2em]  training.south);
                                    \path [line] ([xshift=-2em]  training.south)  --([xshift=-2em]  validation.north);;
                                \end{tikzpicture}
                            };
                        \end{tikzpicture}
                        % \path [line] (mlmodellside) -- (mlmodelrside);
                        \begin{tikzpicture}
                            \node[mlmodelrside] (mlmodelrside) {
                                \begin{tikzpicture}
                                    \node[mlmodelrsidetop, anchor=mlmodelrside.north] (mlmodels) {
                                        \begin{tikzpicture}
                                            \node [mlmodelrsidenode] (ar) {AR};
                                            \node [mlmodelrsidenode, right = of ar] (arima) {ARIMA};
                                            \node [mlmodelrsidenode, right = of arima] (lr) {LR};
                                            \node [mlmodelrsidenode, right = of lr] (svr) {SVR};
                                            \node [mlmodelrsidenode, right = of svr] (cnn) {CNN};
                                            \node [empty, right = of cnn] (ld) {\textbf{\ldots}};
                                            \node [mlmodelrsidenode, right = of ld] (dl) {DL Models};
                                        \end{tikzpicture}
                                    };
                                    \node[mlmodelrsidebottom, anchor=south, below = of mlmodels] (text) {
                                        \vspace{-2\baselineskip}
                                        \begin{itemize}
                                            \item
                                                Exploration of different learning mechanism.
                                            \item
                                                Exploration of different performance evaluation metrics.
                                            \item
                                                Exploration of different Convolutional Neural Network architechture.
                                        \end{itemize}}
                                    % \node[mlmodelrsidebottom, anchor=south, below = of mlmodels] (text) {\newline dsafasfmlkasf \newline \newline}
                                \end{tikzpicture}
                            };
                            % \node [empty, above = of ] (ld) {\scriptsize{Example Text rewrn wejnew rwe rjkne wrkjnewjr kewjkrnewjn rewkj}};
                            % \node [empty] (ld) {\scriptsize{Example Text rewrn wejnew rwe rjkne wrkjnewjr kewjkrnewjn rewkj}};
                        \end{tikzpicture}
                    }
                };
                % \node [block, below = of mlmodel] (train) {Model Training};
                % \node [block, below = of train] (validate) {Model Validation};
                \node [block, below = of mlmodel] (test) {Model Testing};
                \node [block, below = of test] (forecast) {Price Forecasting};
                \node [block, below = of forecast] (evaluate) {Evaluation};
                \node [cloud, left of = forecast, xshift = -2em] (external) {External Factors};
                \node [block, right = 4.5em of evaluate, fill = green!20, text width = 6em] (deploy) {Deployment};
                \node[circle, right of = deploy, xshift = 6em, draw, fill = black!30, font=\tiny] (stop) {STOP};
                % draw edges
                \path [line] (start) -- (data);
                \path [line] (data) -- (preprocess);
                \path [line] (preprocess) -- (features);
                \path [line] (features) -- (mlmodel);
                \path [line] (mlmodel) -- (test);
                % \path [line] (train) -- (validate);
                % \path [line] (validate) -- (test);
                \path [line] (test) -- (forecast);
                \path [line] (forecast) -- (evaluate);
                \path [line] (data) -- node[above] {\scriptsize{Existing standard datasets}} ++(22em,0) |- (mlmodel);
                \path [line] (evaluate) -- node[above] {\scriptsize{Non-Satisfactory}} ++(-22em,0)  |- (preprocess);
                \path [line] (external) -- (forecast);
                \path [line] (evaluate) -- node[above] {\scriptsize{Satisfactory}} (deploy);
                \path [line] (deploy) -- (stop);
            \end{tikzpicture}
        }
        % define caption
        % \caption{Methodology Block Diagram for Machine learning based short-term forecasting of orange and cotton crop prices in the context of the Indian market}
        \caption{Methodology Block Diagram}
        \label{fig: block}
    \end{figure}

    Block schematic of the procedure adopted to carry out is depicted in \ref{fig: block}.

    The procedure adopted to carry out proposed work with details of the procedure is as follows:

    \subsection{Data Collection}
        \subsubsection{Source}
            The dataset is aquired from the Government of India's Open Government Data Platform \cite{data-ogd} and Agricultural Market \cite{data-agm}.
        
        \subsubsection{Period}
            We have taken the data for five years to be able to sufficiently generalize our model and a moderate data size as per our computing infrastructure.
        
    \subsection{Preliminary overview of data}
        The dataset has eight attributes, namely - \texttt{state}, \texttt{district}, \texttt{market}, \texttt{commodity}, \texttt{variety}, \texttt{arrival\_date}, \texttt{min\_price}, \texttt{max\_price}. Our target variable is \texttt{modal\_price}.

        This dataset had some missing values for \texttt{min\_price} and \texttt{max\_price} which we had to pre-process before moving on to building a model based on the dataset.
        
    \subsection{Exploratory Data Analysis}
        \begin{itemize}
            \item 
                Studying the existing Exploratory Data Analysis techniques.
            \item 
                Applying selected techniques on the collected dataset.
            \item
                Based on the performed Exploratory Data Analysis technique generate indicators.
            \item 
                Try to generalize the data based on patterns observed and verify it using indicators.
            \item
                Accordingly make changes in dataset.
        \end{itemize}
    
    \subsection{Preprocessing}
        \begin{itemize}
            \item 
                Imputing the missing values using various strategies like - mean, median, mode or nearest neighbor.
            \item 
                Convert all the categorical values into numerical values by using encoding techniques like one-hot, label, ordinal, etc.
            \item
                Scaling the values to make computation uniform across each iteration and epochs.
            \item
                Normalizing the data points and removing outliers based on statistical tests to avoid overfitting and high model complexity.
        \end{itemize}
    
    \subsection{Feature Engineering}
        \begin{itemize}
            \item
                After using encodings like one-hot encoding our attribute size grows proportional to the distinct values in the original attribute.
            \item
                We use feature selection techniques like Principal Component Analysis to capture the essence of all the attributes and compress the variablity in k-features.
        \end{itemize}    
    
    \subsection{Model Selection}
        \begin{itemize}
            \item
                Study regression based models that can potentially be good in forecasting.
            \item
                Select a subset of these models to form candidate algorithm set.
            \item
                Identify performance evaluation metrics for these implementations.
            \item
                Execute vanilla implementation of each candidate algorithms and select the most promising subset for the next steps.
        \end{itemize}
    
    \subsection{Model Training}
        \begin{itemize}
            \item
                Split the dataset into distinct training and testing set. From the training set keep a subset as validation set.
            \item
                Utilise different strategies for validation like k-fold validation.
            \item
                Cycle between various training validation cycles to tune the parameters to obtain the most optimum set of performance evaluation metrics.
        \end{itemize}
    
    \subsection{ARIMA implementation}
        \begin{itemize}
            \item
                Apply the ARIMA model on the refined dataset.
            \item
                Check all the parameters and tune the parameters to obtain the most optimum set of performance evaluation metrics.
        \end{itemize}
    
        \subsection{Model Evaluation}
        \begin{itemize}
            \item
                Apply all the trained models on the testing set.
            \item
                Select the best model based on all the performance evaluation metrics. 
            \item
                The best performing model will be our final model for forecasting.
        \end{itemize}
